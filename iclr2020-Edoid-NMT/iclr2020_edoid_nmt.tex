
\documentclass{article} % For LaTeX2e
\usepackage{iclr2020_conference,times}

% Optional math commands from https://github.com/goodfeli/dlbook_notation.
%%%%% NEW MATH DEFINITIONS %%%%%

\usepackage{amsmath,amsfonts,bm}

% Mark sections of captions for referring to divisions of figures
\newcommand{\figleft}{{\em (Left)}}
\newcommand{\figcenter}{{\em (Center)}}
\newcommand{\figright}{{\em (Right)}}
\newcommand{\figtop}{{\em (Top)}}
\newcommand{\figbottom}{{\em (Bottom)}}
\newcommand{\captiona}{{\em (a)}}
\newcommand{\captionb}{{\em (b)}}
\newcommand{\captionc}{{\em (c)}}
\newcommand{\captiond}{{\em (d)}}

% Highlight a newly defined term
\newcommand{\newterm}[1]{{\bf #1}}


% Figure reference, lower-case.
\def\figref#1{figure~\ref{#1}}
% Figure reference, capital. For start of sentence
\def\Figref#1{Figure~\ref{#1}}
\def\twofigref#1#2{figures \ref{#1} and \ref{#2}}
\def\quadfigref#1#2#3#4{figures \ref{#1}, \ref{#2}, \ref{#3} and \ref{#4}}
% Section reference, lower-case.
\def\secref#1{section~\ref{#1}}
% Section reference, capital.
\def\Secref#1{Section~\ref{#1}}
% Reference to two sections.
\def\twosecrefs#1#2{sections \ref{#1} and \ref{#2}}
% Reference to three sections.
\def\secrefs#1#2#3{sections \ref{#1}, \ref{#2} and \ref{#3}}
% Reference to an equation, lower-case.
\def\eqref#1{equation~\ref{#1}}
% Reference to an equation, upper case
\def\Eqref#1{Equation~\ref{#1}}
% A raw reference to an equation---avoid using if possible
\def\plaineqref#1{\ref{#1}}
% Reference to a chapter, lower-case.
\def\chapref#1{chapter~\ref{#1}}
% Reference to an equation, upper case.
\def\Chapref#1{Chapter~\ref{#1}}
% Reference to a range of chapters
\def\rangechapref#1#2{chapters\ref{#1}--\ref{#2}}
% Reference to an algorithm, lower-case.
\def\algref#1{algorithm~\ref{#1}}
% Reference to an algorithm, upper case.
\def\Algref#1{Algorithm~\ref{#1}}
\def\twoalgref#1#2{algorithms \ref{#1} and \ref{#2}}
\def\Twoalgref#1#2{Algorithms \ref{#1} and \ref{#2}}
% Reference to a part, lower case
\def\partref#1{part~\ref{#1}}
% Reference to a part, upper case
\def\Partref#1{Part~\ref{#1}}
\def\twopartref#1#2{parts \ref{#1} and \ref{#2}}

\def\ceil#1{\lceil #1 \rceil}
\def\floor#1{\lfloor #1 \rfloor}
\def\1{\bm{1}}
\newcommand{\train}{\mathcal{D}}
\newcommand{\valid}{\mathcal{D_{\mathrm{valid}}}}
\newcommand{\test}{\mathcal{D_{\mathrm{test}}}}

\def\eps{{\epsilon}}


% Random variables
\def\reta{{\textnormal{$\eta$}}}
\def\ra{{\textnormal{a}}}
\def\rb{{\textnormal{b}}}
\def\rc{{\textnormal{c}}}
\def\rd{{\textnormal{d}}}
\def\re{{\textnormal{e}}}
\def\rf{{\textnormal{f}}}
\def\rg{{\textnormal{g}}}
\def\rh{{\textnormal{h}}}
\def\ri{{\textnormal{i}}}
\def\rj{{\textnormal{j}}}
\def\rk{{\textnormal{k}}}
\def\rl{{\textnormal{l}}}
% rm is already a command, just don't name any random variables m
\def\rn{{\textnormal{n}}}
\def\ro{{\textnormal{o}}}
\def\rp{{\textnormal{p}}}
\def\rq{{\textnormal{q}}}
\def\rr{{\textnormal{r}}}
\def\rs{{\textnormal{s}}}
\def\rt{{\textnormal{t}}}
\def\ru{{\textnormal{u}}}
\def\rv{{\textnormal{v}}}
\def\rw{{\textnormal{w}}}
\def\rx{{\textnormal{x}}}
\def\ry{{\textnormal{y}}}
\def\rz{{\textnormal{z}}}

% Random vectors
\def\rvepsilon{{\mathbf{\epsilon}}}
\def\rvtheta{{\mathbf{\theta}}}
\def\rva{{\mathbf{a}}}
\def\rvb{{\mathbf{b}}}
\def\rvc{{\mathbf{c}}}
\def\rvd{{\mathbf{d}}}
\def\rve{{\mathbf{e}}}
\def\rvf{{\mathbf{f}}}
\def\rvg{{\mathbf{g}}}
\def\rvh{{\mathbf{h}}}
\def\rvu{{\mathbf{i}}}
\def\rvj{{\mathbf{j}}}
\def\rvk{{\mathbf{k}}}
\def\rvl{{\mathbf{l}}}
\def\rvm{{\mathbf{m}}}
\def\rvn{{\mathbf{n}}}
\def\rvo{{\mathbf{o}}}
\def\rvp{{\mathbf{p}}}
\def\rvq{{\mathbf{q}}}
\def\rvr{{\mathbf{r}}}
\def\rvs{{\mathbf{s}}}
\def\rvt{{\mathbf{t}}}
\def\rvu{{\mathbf{u}}}
\def\rvv{{\mathbf{v}}}
\def\rvw{{\mathbf{w}}}
\def\rvx{{\mathbf{x}}}
\def\rvy{{\mathbf{y}}}
\def\rvz{{\mathbf{z}}}

% Elements of random vectors
\def\erva{{\textnormal{a}}}
\def\ervb{{\textnormal{b}}}
\def\ervc{{\textnormal{c}}}
\def\ervd{{\textnormal{d}}}
\def\erve{{\textnormal{e}}}
\def\ervf{{\textnormal{f}}}
\def\ervg{{\textnormal{g}}}
\def\ervh{{\textnormal{h}}}
\def\ervi{{\textnormal{i}}}
\def\ervj{{\textnormal{j}}}
\def\ervk{{\textnormal{k}}}
\def\ervl{{\textnormal{l}}}
\def\ervm{{\textnormal{m}}}
\def\ervn{{\textnormal{n}}}
\def\ervo{{\textnormal{o}}}
\def\ervp{{\textnormal{p}}}
\def\ervq{{\textnormal{q}}}
\def\ervr{{\textnormal{r}}}
\def\ervs{{\textnormal{s}}}
\def\ervt{{\textnormal{t}}}
\def\ervu{{\textnormal{u}}}
\def\ervv{{\textnormal{v}}}
\def\ervw{{\textnormal{w}}}
\def\ervx{{\textnormal{x}}}
\def\ervy{{\textnormal{y}}}
\def\ervz{{\textnormal{z}}}

% Random matrices
\def\rmA{{\mathbf{A}}}
\def\rmB{{\mathbf{B}}}
\def\rmC{{\mathbf{C}}}
\def\rmD{{\mathbf{D}}}
\def\rmE{{\mathbf{E}}}
\def\rmF{{\mathbf{F}}}
\def\rmG{{\mathbf{G}}}
\def\rmH{{\mathbf{H}}}
\def\rmI{{\mathbf{I}}}
\def\rmJ{{\mathbf{J}}}
\def\rmK{{\mathbf{K}}}
\def\rmL{{\mathbf{L}}}
\def\rmM{{\mathbf{M}}}
\def\rmN{{\mathbf{N}}}
\def\rmO{{\mathbf{O}}}
\def\rmP{{\mathbf{P}}}
\def\rmQ{{\mathbf{Q}}}
\def\rmR{{\mathbf{R}}}
\def\rmS{{\mathbf{S}}}
\def\rmT{{\mathbf{T}}}
\def\rmU{{\mathbf{U}}}
\def\rmV{{\mathbf{V}}}
\def\rmW{{\mathbf{W}}}
\def\rmX{{\mathbf{X}}}
\def\rmY{{\mathbf{Y}}}
\def\rmZ{{\mathbf{Z}}}

% Elements of random matrices
\def\ermA{{\textnormal{A}}}
\def\ermB{{\textnormal{B}}}
\def\ermC{{\textnormal{C}}}
\def\ermD{{\textnormal{D}}}
\def\ermE{{\textnormal{E}}}
\def\ermF{{\textnormal{F}}}
\def\ermG{{\textnormal{G}}}
\def\ermH{{\textnormal{H}}}
\def\ermI{{\textnormal{I}}}
\def\ermJ{{\textnormal{J}}}
\def\ermK{{\textnormal{K}}}
\def\ermL{{\textnormal{L}}}
\def\ermM{{\textnormal{M}}}
\def\ermN{{\textnormal{N}}}
\def\ermO{{\textnormal{O}}}
\def\ermP{{\textnormal{P}}}
\def\ermQ{{\textnormal{Q}}}
\def\ermR{{\textnormal{R}}}
\def\ermS{{\textnormal{S}}}
\def\ermT{{\textnormal{T}}}
\def\ermU{{\textnormal{U}}}
\def\ermV{{\textnormal{V}}}
\def\ermW{{\textnormal{W}}}
\def\ermX{{\textnormal{X}}}
\def\ermY{{\textnormal{Y}}}
\def\ermZ{{\textnormal{Z}}}

% Vectors
\def\vzero{{\bm{0}}}
\def\vone{{\bm{1}}}
\def\vmu{{\bm{\mu}}}
\def\vtheta{{\bm{\theta}}}
\def\va{{\bm{a}}}
\def\vb{{\bm{b}}}
\def\vc{{\bm{c}}}
\def\vd{{\bm{d}}}
\def\ve{{\bm{e}}}
\def\vf{{\bm{f}}}
\def\vg{{\bm{g}}}
\def\vh{{\bm{h}}}
\def\vi{{\bm{i}}}
\def\vj{{\bm{j}}}
\def\vk{{\bm{k}}}
\def\vl{{\bm{l}}}
\def\vm{{\bm{m}}}
\def\vn{{\bm{n}}}
\def\vo{{\bm{o}}}
\def\vp{{\bm{p}}}
\def\vq{{\bm{q}}}
\def\vr{{\bm{r}}}
\def\vs{{\bm{s}}}
\def\vt{{\bm{t}}}
\def\vu{{\bm{u}}}
\def\vv{{\bm{v}}}
\def\vw{{\bm{w}}}
\def\vx{{\bm{x}}}
\def\vy{{\bm{y}}}
\def\vz{{\bm{z}}}

% Elements of vectors
\def\evalpha{{\alpha}}
\def\evbeta{{\beta}}
\def\evepsilon{{\epsilon}}
\def\evlambda{{\lambda}}
\def\evomega{{\omega}}
\def\evmu{{\mu}}
\def\evpsi{{\psi}}
\def\evsigma{{\sigma}}
\def\evtheta{{\theta}}
\def\eva{{a}}
\def\evb{{b}}
\def\evc{{c}}
\def\evd{{d}}
\def\eve{{e}}
\def\evf{{f}}
\def\evg{{g}}
\def\evh{{h}}
\def\evi{{i}}
\def\evj{{j}}
\def\evk{{k}}
\def\evl{{l}}
\def\evm{{m}}
\def\evn{{n}}
\def\evo{{o}}
\def\evp{{p}}
\def\evq{{q}}
\def\evr{{r}}
\def\evs{{s}}
\def\evt{{t}}
\def\evu{{u}}
\def\evv{{v}}
\def\evw{{w}}
\def\evx{{x}}
\def\evy{{y}}
\def\evz{{z}}

% Matrix
\def\mA{{\bm{A}}}
\def\mB{{\bm{B}}}
\def\mC{{\bm{C}}}
\def\mD{{\bm{D}}}
\def\mE{{\bm{E}}}
\def\mF{{\bm{F}}}
\def\mG{{\bm{G}}}
\def\mH{{\bm{H}}}
\def\mI{{\bm{I}}}
\def\mJ{{\bm{J}}}
\def\mK{{\bm{K}}}
\def\mL{{\bm{L}}}
\def\mM{{\bm{M}}}
\def\mN{{\bm{N}}}
\def\mO{{\bm{O}}}
\def\mP{{\bm{P}}}
\def\mQ{{\bm{Q}}}
\def\mR{{\bm{R}}}
\def\mS{{\bm{S}}}
\def\mT{{\bm{T}}}
\def\mU{{\bm{U}}}
\def\mV{{\bm{V}}}
\def\mW{{\bm{W}}}
\def\mX{{\bm{X}}}
\def\mY{{\bm{Y}}}
\def\mZ{{\bm{Z}}}
\def\mBeta{{\bm{\beta}}}
\def\mPhi{{\bm{\Phi}}}
\def\mLambda{{\bm{\Lambda}}}
\def\mSigma{{\bm{\Sigma}}}

% Tensor
\DeclareMathAlphabet{\mathsfit}{\encodingdefault}{\sfdefault}{m}{sl}
\SetMathAlphabet{\mathsfit}{bold}{\encodingdefault}{\sfdefault}{bx}{n}
\newcommand{\tens}[1]{\bm{\mathsfit{#1}}}
\def\tA{{\tens{A}}}
\def\tB{{\tens{B}}}
\def\tC{{\tens{C}}}
\def\tD{{\tens{D}}}
\def\tE{{\tens{E}}}
\def\tF{{\tens{F}}}
\def\tG{{\tens{G}}}
\def\tH{{\tens{H}}}
\def\tI{{\tens{I}}}
\def\tJ{{\tens{J}}}
\def\tK{{\tens{K}}}
\def\tL{{\tens{L}}}
\def\tM{{\tens{M}}}
\def\tN{{\tens{N}}}
\def\tO{{\tens{O}}}
\def\tP{{\tens{P}}}
\def\tQ{{\tens{Q}}}
\def\tR{{\tens{R}}}
\def\tS{{\tens{S}}}
\def\tT{{\tens{T}}}
\def\tU{{\tens{U}}}
\def\tV{{\tens{V}}}
\def\tW{{\tens{W}}}
\def\tX{{\tens{X}}}
\def\tY{{\tens{Y}}}
\def\tZ{{\tens{Z}}}


% Graph
\def\gA{{\mathcal{A}}}
\def\gB{{\mathcal{B}}}
\def\gC{{\mathcal{C}}}
\def\gD{{\mathcal{D}}}
\def\gE{{\mathcal{E}}}
\def\gF{{\mathcal{F}}}
\def\gG{{\mathcal{G}}}
\def\gH{{\mathcal{H}}}
\def\gI{{\mathcal{I}}}
\def\gJ{{\mathcal{J}}}
\def\gK{{\mathcal{K}}}
\def\gL{{\mathcal{L}}}
\def\gM{{\mathcal{M}}}
\def\gN{{\mathcal{N}}}
\def\gO{{\mathcal{O}}}
\def\gP{{\mathcal{P}}}
\def\gQ{{\mathcal{Q}}}
\def\gR{{\mathcal{R}}}
\def\gS{{\mathcal{S}}}
\def\gT{{\mathcal{T}}}
\def\gU{{\mathcal{U}}}
\def\gV{{\mathcal{V}}}
\def\gW{{\mathcal{W}}}
\def\gX{{\mathcal{X}}}
\def\gY{{\mathcal{Y}}}
\def\gZ{{\mathcal{Z}}}

% Sets
\def\sA{{\mathbb{A}}}
\def\sB{{\mathbb{B}}}
\def\sC{{\mathbb{C}}}
\def\sD{{\mathbb{D}}}
% Don't use a set called E, because this would be the same as our symbol
% for expectation.
\def\sF{{\mathbb{F}}}
\def\sG{{\mathbb{G}}}
\def\sH{{\mathbb{H}}}
\def\sI{{\mathbb{I}}}
\def\sJ{{\mathbb{J}}}
\def\sK{{\mathbb{K}}}
\def\sL{{\mathbb{L}}}
\def\sM{{\mathbb{M}}}
\def\sN{{\mathbb{N}}}
\def\sO{{\mathbb{O}}}
\def\sP{{\mathbb{P}}}
\def\sQ{{\mathbb{Q}}}
\def\sR{{\mathbb{R}}}
\def\sS{{\mathbb{S}}}
\def\sT{{\mathbb{T}}}
\def\sU{{\mathbb{U}}}
\def\sV{{\mathbb{V}}}
\def\sW{{\mathbb{W}}}
\def\sX{{\mathbb{X}}}
\def\sY{{\mathbb{Y}}}
\def\sZ{{\mathbb{Z}}}

% Entries of a matrix
\def\emLambda{{\Lambda}}
\def\emA{{A}}
\def\emB{{B}}
\def\emC{{C}}
\def\emD{{D}}
\def\emE{{E}}
\def\emF{{F}}
\def\emG{{G}}
\def\emH{{H}}
\def\emI{{I}}
\def\emJ{{J}}
\def\emK{{K}}
\def\emL{{L}}
\def\emM{{M}}
\def\emN{{N}}
\def\emO{{O}}
\def\emP{{P}}
\def\emQ{{Q}}
\def\emR{{R}}
\def\emS{{S}}
\def\emT{{T}}
\def\emU{{U}}
\def\emV{{V}}
\def\emW{{W}}
\def\emX{{X}}
\def\emY{{Y}}
\def\emZ{{Z}}
\def\emSigma{{\Sigma}}

% entries of a tensor
% Same font as tensor, without \bm wrapper
\newcommand{\etens}[1]{\mathsfit{#1}}
\def\etLambda{{\etens{\Lambda}}}
\def\etA{{\etens{A}}}
\def\etB{{\etens{B}}}
\def\etC{{\etens{C}}}
\def\etD{{\etens{D}}}
\def\etE{{\etens{E}}}
\def\etF{{\etens{F}}}
\def\etG{{\etens{G}}}
\def\etH{{\etens{H}}}
\def\etI{{\etens{I}}}
\def\etJ{{\etens{J}}}
\def\etK{{\etens{K}}}
\def\etL{{\etens{L}}}
\def\etM{{\etens{M}}}
\def\etN{{\etens{N}}}
\def\etO{{\etens{O}}}
\def\etP{{\etens{P}}}
\def\etQ{{\etens{Q}}}
\def\etR{{\etens{R}}}
\def\etS{{\etens{S}}}
\def\etT{{\etens{T}}}
\def\etU{{\etens{U}}}
\def\etV{{\etens{V}}}
\def\etW{{\etens{W}}}
\def\etX{{\etens{X}}}
\def\etY{{\etens{Y}}}
\def\etZ{{\etens{Z}}}

% The true underlying data generating distribution
\newcommand{\pdata}{p_{\rm{data}}}
% The empirical distribution defined by the training set
\newcommand{\ptrain}{\hat{p}_{\rm{data}}}
\newcommand{\Ptrain}{\hat{P}_{\rm{data}}}
% The model distribution
\newcommand{\pmodel}{p_{\rm{model}}}
\newcommand{\Pmodel}{P_{\rm{model}}}
\newcommand{\ptildemodel}{\tilde{p}_{\rm{model}}}
% Stochastic autoencoder distributions
\newcommand{\pencode}{p_{\rm{encoder}}}
\newcommand{\pdecode}{p_{\rm{decoder}}}
\newcommand{\precons}{p_{\rm{reconstruct}}}

\newcommand{\laplace}{\mathrm{Laplace}} % Laplace distribution

\newcommand{\E}{\mathbb{E}}
\newcommand{\Ls}{\mathcal{L}}
\newcommand{\R}{\mathbb{R}}
\newcommand{\emp}{\tilde{p}}
\newcommand{\lr}{\alpha}
\newcommand{\reg}{\lambda}
\newcommand{\rect}{\mathrm{rectifier}}
\newcommand{\softmax}{\mathrm{softmax}}
\newcommand{\sigmoid}{\sigma}
\newcommand{\softplus}{\zeta}
\newcommand{\KL}{D_{\mathrm{KL}}}
\newcommand{\Var}{\mathrm{Var}}
\newcommand{\standarderror}{\mathrm{SE}}
\newcommand{\Cov}{\mathrm{Cov}}
% Wolfram Mathworld says $L^2$ is for function spaces and $\ell^2$ is for vectors
% But then they seem to use $L^2$ for vectors throughout the site, and so does
% wikipedia.
\newcommand{\normlzero}{L^0}
\newcommand{\normlone}{L^1}
\newcommand{\normltwo}{L^2}
\newcommand{\normlp}{L^p}
\newcommand{\normmax}{L^\infty}

\newcommand{\parents}{Pa} % See usage in notation.tex. Chosen to match Daphne's book.

\DeclareMathOperator*{\argmax}{arg\,max}
\DeclareMathOperator*{\argmin}{arg\,min}

\DeclareMathOperator{\sign}{sign}
\DeclareMathOperator{\Tr}{Tr}
\let\ab\allowbreak


\usepackage{hyperref}
\usepackage{url}
\usepackage{graphicx}
\usepackage{booktabs}
\usepackage{multirow}

\def\UrlBreaks{\do\/\do-}

\title{Towards Neural Machine Translation \\ for Edoid Languages}

% Authors must not appear in the submitted version. They should be hidden
% as long as the \iclrfinalcopy macro remains commented out below.
% Non-anonymous submissions will be rejected without review.

\author{Iroro Fred \d{\`O}n\d{\`o}m\d{\`e} Orife \\
Niger-Volta Language Technologies Institute\\
\texttt{iroro@alumni.cmu.edu} \\
%\And
%Dr. John N. Orife\\
%Indiana University of Pennsylvania, \\
%\texttt{jorife@iup.edu} \\
}

% The \author macro works with any number of authors. There are two commands
% used to separate the names and addresses of multiple authors: \And and \AND.
%
% Using \And between authors leaves it to \LaTeX{} to determine where to break
% the lines. Using \AND forces a linebreak at that point. So, if \LaTeX{}
% puts 3 of 4 authors names on the first line, and the last on the second
% line, try using \AND instead of \And before the third author name.

\newcommand{\fix}{\marginpar{FIX}}
\newcommand{\new}{\marginpar{NEW}}

%\iclrfinalcopy % Uncomment for camera-ready version, but NOT for submission.
\begin{document}


\maketitle

%\begin{abstract}
%Many Nigerian languages have   Using the new JW300 public dataset, we trained and evaluated baseline translation models for four widely spoken languages: \d{\`E}d{\'o}, \d{\`E}s{\'a}n, Urhobo and Isoko. Trained models, code and datasets have been open-sourced to advance future research efforts on Edoid language technology.\\

%Given the empowering potential of language technology, we investigate the feasibility of Neural Machine Translation (NMT) for speakers of Edoid languages of Southern Nigeria.
%\end{abstract}

\section{Introduction}

% How language inequality came about
Many of the over 500 languages are spoken in Nigeria today have relinquished their previous prestige and purpose in modern society to English and Nigerian Pidgin, notably amongst the younger generations. Unlike numerous East and South Asian societies, which preserved the socio-linguistic status of their indigenous languages under colonial rule, communities with primarily oral traditions have been the most susceptible to language endangerment \citep{rolle2013phonetics, omo2004esan}.

% How language inequality affects people and indigenous language research
For tens of millions of speakers, language inequalities manifest themselves as unequal access to information, communications, health care, security along with attenuated participation in political and civic life. These inequities are further exacerbated in a technological age, where only the most highly resourced (i.e. colonial) languages become the milieu for economic advancement \citep{odojelanguage, awobuluyi201626, ganagana2019contrastive}. Finally, there have been practical and technical challenges in language technology for indigenous languages like orthographic standardizations and consistent diacritic representation (Unicode) in electronic media and across device types. 

% How can NMT help?
For almost-extinct languages, machine translation offers hope for language documentation and preservation. For speakers of minority Nigerian languages, it can facilitate good governance, national development and offers a path for technological, economic, social and political participation and empowerment to those with unequal access \citep{odoje201612, odojelanguage}. Using the new JW300 public dataset, we trained and evaluated baseline Neural Machine Translation (NMT) models for four widely spoken Edoid languages: \d{\`E}d{\'o}, \d{\`E}s{\'a}n, Urhobo and Isoko. 

% based on economic status, demography, location and language
%From a research vantage, these languages have been labelled ``low resource" due to the scarcity of academic research, online-datasets, funding and interest, due to perceptions about the prestige and utility of these languages in contemporary African life  
% Among other factors, the scarcity of highly esteemed works of literary scholarship has worked to devalue indigenous languages, making them susceptible to being replaced by other languages of power and status \citep{awobuluyi201626}. 

\section{Edoid Languages}

% Population
% Edo: 2.5M \\
% Esan: 500k \\
% Urhobo: 500k - 1.5M/2M
% Isoko: 658,000 

Belonging to the eastern sub-branch of the Volta-Niger family within the Niger-Congo phylum, and spoken by approximately 5 million people, the Edoid languages of Southern Nigeria (Edo and Delta states) comprise over two dozen so-called ``minority" languages. The term \emph{Edoid} stems from \d{\`E}d{\'o}, the most broadly spoken member langauge and the language of the famed Kingdom of Benin. \d{\`E}d{\'o}, \d{\`E}s{\'a}n are members of the North-Central branch while Urhobo and Isoko belong to the South-Western family \citep{ethnologue_2019}. These languages were selected based on the availability of text and because they are the most widely spoken.

Edoid langauges generally employ the SVO constituent order type, open syllable systems with very few consonant clusters. Each language has at least two basic tone levels, high (H) and low (L) with kinetic, downstepped or contour tones variously utilized. As tone patterns serve different lexical and grammatical functions, ``the phonetic and phonological implementation of this system is in fact complex and difficult to pin down" \citep{rolle2013phonetics, ogie2009multi, adeniyi2010tone, ilolo2013vowel}. Finally, nasalisation is very common for both vowels and consonants \citep{Elugbe_1989, isoko_phonetics, ikoyo2018phonetic}. 

% urban  centers  of  Urhoboland  such  as  Effurun,  Sapele,  Ughelli,  and  Warri  do  not  use  and/or  speak  the  language. 
%For these reasons, speaker populatioin greater than acutally speakers, and low usage in the young, we consider one should consider the Edoid languages as highly endangered, inspite a large speaker population \citep{rolle2013phonetics}.
%A language can only resist death/extinction if it is able to move from the status of oracy to awritten status. Okojie (1994) 

Within Nigeria there is scholarship on rule, phrase and statistical machine translation systems for majority tongues of Yor{\`u}b{\'a}, Igbo and Hausa \citep{odojelanguage}. The present study is the first work known to the authors done in computational linguistics for any of the Edoid langauges, specifically for machine translation.


\section{Methodology}
\label{methods}

We first built baseline models using the Transformer architecture, the dominant modeling approach for NMT. The Transformer uses an encoder-decoder structure with stacked multi-head self-attention and fully connected layers \citep{NIPS2017_7181}. Given the performance of Byte Pair Encoding (BPE) subword tokenization for low-resourced South African languages, and the size of our datasets, we trained baseline models based on the ablation study results by Martinus et al., some 4000 BPE tokens \citep{focus_southafrica}. Models were then re-trained for all four languages using the standard word-level tokenization.

\paragraph{Dataset:} The recently published JW300 dataset is a large-scale, parallel corpus for Machine Translation (MT) comprising more than three hundred languages of which 101 are African \citep{agic-vulic-2019-jw300}. JW300 text is drawn from the Watchtower and Awake! religious magazines by Jehovah's Witnesses (JW). The test set contains sentences with the highest coverage across all other languages in the corpus. The relative training set cardinality is listed in Appendix Table~\ref{results}. 

\paragraph{Models:} The open-source, Python 3 machine translation toolkit \texttt{JoeyNMT} was used to train Transformer models \citep{JoeyNMT}. Our training hardware was the commodity free-tier configuration on Google Colaboratory, a single core Xeon CPU instance and a Tesla K80 GPU. Model training elapsed over multiple days, as experiments were repeated for the different tokenizations.

\section{Results}
\label{results}

% Dad's view on the concepts 
\paragraph{Qualitative:} Urhobo and Isoko with larger training texts unsurprisingly had higher BLEU scores which generally correlated with the translation quality when reviewed by L1 speakers. BPE tokenization provided approximately a 37\% boost across dev and test sets for \d{\`E}d{\'o} and \d{\`E}s{\'a}n, a 32\% boost for Urhobo but was flat to slightly worse than word-level tokenization for Isoko. Full scores and examples are listed in the Appendix.

\paragraph{Error Analysis:} While performing error analyses on the model predictions, we observed predictions that included  dataset requires more preprocessing to remove scriptural chapter and verse text.  chapters\/verse names and figures. This will make the model more generally useful outside of religious text translations.

\section{Future Work and Conclusions}
Fertile avenues for future work include investigating back-translation and different (subword) tokenization approaches as well as specific consideration of linguistic knowledge. We hope this initial effort will assist translators, bootstrap development and sustenance of scholarly and literary traditions and energize academic and industry interest in language technology for socio-linguistic and economic empowerment. Ultimately, languages with predominantly oral traditions will most benefit from language technologies with direct audio-speech interfaces. This present work is but one step towards that goal. All public-domain datasets referenced in this work are available on GitHub.\footnote{\url{https://github.com/Niger-Volta-LTI}}
 
% Thee include social justice by addressing an aspect of technological language inequality, language preservation and by establishing baselines and from which to build on. Given the comparatively low (Oladele Awobuluyi) litearay traditions but the very strong oral traditions, foundational language technologies based on good clean text, like language and translation models are just the start, but very important precusor to speech interfaces. Imagine a world in which a culture rooted in a strong oral tradition can make use of Speech-to-Speech interfaces, speaking and being spoken to idiomatically. This is where the future of African langauge technology lies and mahcine translation and good clean datasets are the core.   
% 


% Applications to film and cinema technology to automatically subtitle and translate Edoid language films.

% paint a future where robust speech-to-speech interfaces kick ass and allow everyone to speak freely




\subsubsection*{Acknowledgments}
The authors thank Dr. Ajovi B. Scott-Emuakpor, MD and Dr. John Nevboyeri Orife for their encouragement, support and qualitative critiques of the translations.

\bibliography{iclr2020_conference}
\bibliographystyle{iclr2020_conference}

\clearpage

\appendix
\section{Appendix}

\begin{table}[h]
\caption{Per-language BLEU scores by BPE or word-level tokenization}
\label{results}
\begin{center}
\begin{tabular}{c@{\qquad}ccc@{\qquad}ccc}
  \toprule
  \multirow{2}{*}{\raisebox{-\heavyrulewidth}{\textbf{Language}}} & \multicolumn{2}{c}{\textbf{BPE}} & \multicolumn{2}{c}{\textbf{Word}} & \multirow{2}{*}{\raisebox{-\heavyrulewidth}{\textbf{Tokens}}} & \multirow{2}{*}{\raisebox{-\heavyrulewidth}{\textbf{Sentences}}}
  	 \\
  \cmidrule{2-5}
  & dev & test & dev & test \\
  \midrule
  \d{\`E}d{\'o}  & 7.92 & 12.49 & 5.99 & 8.24 &  229,307 & 10,188 \\
  \d{\`E}s{\'a}n & 4.94 & 6.25 & 3.39 & 5.30 & 87,025 & 4,128 \\
    \midrule
  Urhobo  & 15.91 & 28.82 & 11.80 & 22.39 & 519,981 & 25,610 \\
  Isoko   & 32.58 & 38.05 & 32.38 & 38.91 & 4,824,998 & 214,546 \\
  \bottomrule
  \end{tabular}
\end{center}
\end{table}

\begin{table}[h]
\caption{Example Translations}
\label{translations}
\begin{center}
  \begin{tabular}{ll}
     \textbf{\d{\`E}d{\'o}}  & \\
     \midrule
     \midrule
     Source:   &  Reading and meditating on real - life Bible accounts can help us to do what ?  \\
Reference: & De vbene okha ni rre Baibol ya ru iyob\d{o} ne ima h\d{e} ?  \\
Hypothesis: & De emwi ne ima gha ru ne ima mieke na gha mw\d{e} ir\d{e}nmwi n\d{o} gbae vbekpae Jehova ?  \\
     \midrule
	Source:   &    What are the rewards for being humble ? \\
	Reference:  &  Ma ghaa mu egbe rriot\d{o} , de afiangbe na lae mi\d{e}n ? \\
	Prediction:  & De emwi n\d{o} kh\d{e}ke ne \d{o}mwa n\d{o} dizigha \d{o}y\d{e}vbu ru ? \\
	 \bottomrule
     \\
    \textbf{\d{\`E}s{\'a}n}  & \\
     \midrule
     \midrule
	Source:      &  I WAS raised in Graz , Austria . \\
	Reference:    & AGBA\d{E}BHO nati\d{o}le Graz bhi Austria , \d{o}le m\d{e}n da wanre . \\
	Prediction:   & M\d{e}n da ha khian \d{o}ne isikulu , m\d{e}n da d\d{o} ha khian \d{o}ne isikulu . \\
     \midrule
     Source:    &   We should also strive to help others spiritually . \\
	 Reference:  &  Ahami\d{e}n mhan re \d{e}ghe bhi ot\d{o} r\d{e} ha lu\d{e} iBaibo , \\
	 Prediction:  &  Mhan d\d{e} sab\d{o} r\d{e}kpa mhan r\d{e} sab\d{o} ha mh\d{o}n ur\d{e}\d{o}bh\d{o} b\d{o}si eria . \\
	 \bottomrule
	\\
    \textbf{Urhobo}  & \\
     \midrule
     \midrule
	Source:    &                   But freedom from what ? \\
	Reference: &  		   Ẹk\d{e}vu\d{o}vo , \d{e}dia v\d{o} yen egbomọph\d{e} na che si ayen nu ? \\
	Prediction: & ( 1 Pita 3 : 1 ) Ẹk\d{e}vu\d{o}vo , die yen egbom\d{o}ph\d{e}  \\
     \midrule

	Source:    &   Today he is serving at Bethel . \\ 
	Reference:  &  Nonẹna , \d{o} ga vw\d{e} B\d{e}t\d{e}l .\\
	Prediction: &  Nonẹna , \d{o} ga vw\d{e} B\d{e}t\d{e}l asaọkiephana . \\
	\bottomrule
	\\
    \textbf{Isoko}  & \\
     \midrule
     \midrule
	Source:      & Still , words of apology are a strong force toward making peace . \\
	Reference:   & Ghele na , eme unu - uwou u re fi ob\d{o} h\d{o} gaga eva\d{o} eruo udhedh\d{e} .\\
	Prediction:  & Ghele na , eme unu - uwou y\d{o} \d{e}gba ologbo n\d{o} ma re ro ru udhedh\d{e} .\\
  \midrule
	 Source:   & We can even ask God to ‘ create in us a pure heart . ’ \\
	Reference:  & Ma r\d{e} sae tub\d{e} yare \d{O}gh\d{e}n\d{e} re \d{o}  ‘ kẹ omai eva efuafo . ’ \\
	Prediction: & Ma r\d{e} sae tub\d{e} yare \d{O}gh\d{e}n\d{e} re \d{o}  ‘ ma omai eva efuafo .  \\
	 \bottomrule
     
  \end{tabular}
\end{center}
\end{table}

\end{document}
